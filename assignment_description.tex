\documentclass{article}

% Language setting
% Replace `english' with e.g. `spanish' to change the document language
\usepackage[english]{babel}

% Set page size and margins
% Replace `letterpaper' with`a4paper' for UK/EU standard size
\usepackage[letterpaper,top=2cm,bottom=2cm,left=3cm,right=3cm,marginparwidth=1.75cm]{geometry}

% Useful packages
\usepackage{amsmath}
\usepackage{graphicx}
\usepackage[colorlinks=true, allcolors=blue]{hyperref}

\title{Coding Assignment 2}
\author{EC 400}

\begin{document}
\maketitle



\noindent {\bf Setup. }
Unzip the base code provided on blackboard into a new directory. You can check that everything is as it should be by running
\begin{verbatim} 
python3 gridworld.py -g MazeGrid
\end{verbatim} 
You should see the agent move around randomly on the gridworld until it reaches a termination step. 

\bigskip

\noindent {\em Hint:} In Anaconda Spyder, under "Run" you can select "Configuration per File" and enter the command line options, i.e., "-g MazeGrid". If you are not using command line to write Python, whatever software you are using should have a similar option. 

\bigskip

\noindent {\em Note:} Spyder occasionally freezes after the above program terminates. I usually reset it by restarting the kernel ("CONTROL" + "."). 


\bigskip

\noindent {\bf Q-learning:} Write code to do Q-learning on gridworld by editing the  qlearningAgents.py file. In that file, 
fill in the code for the functions 
getQValue, ComputeValuefromQvalues, computeActionFromQValues, getAction, and update. 

Make sure that in your computeValueFromQValues and computeActionFromQValues functions, you only access Q values by calling getQValue.

Once you fill in your own code, remove the        util.raiseNotDefined() commands

\bigskip

Once you are done, you can run 

\begin{verbatim} 
python3 gridworld.py -a q -k 15
\end{verbatim} to see the outcome of 15 episodes of Q-learning. 
\end{verbatim}



\end{document}